%! tex root = ../master.tex

\chapter{Analysis}
\todo[inline,color=green]{Intro for te analysis.}


\section{Real-time systems}

A real-time system is handling data as it comes in. Besides the usual computations needed to be done, 
the time until a result generated is also important. 
This means that the applications are scheduled in a manner that would allow them to respond quick and predictable. 
The real-time systems were developed in order to answer the requirements of different fields of work where time is critical: military, automotive or aeronautics.

\section{Integrated Modular Avionics}

In the aeronautic industry, the devices and peripherals used for controlling the machines have to be reliable and operate using the necessary time constraints. Integrated Modular Avionics stands for 
a network of real-time modules deployed for use in the avionics industry. 
The main characteristics of such a network are: modularity and keeping track of the time and criticality levels. 
This implies that the applications running on a system share the resources following a set of well defined rules. 
There are guidelines of how such a system should be designed and implemented, 
as well as standards that specify how space and time should be partitioned.

\section{ARINC 653}
\subsection{Overview}
\todo[inline,color=green]{ARINC 653 is a standard that was originally designed and created for the avionics industry.
It has been created to work within Integrated Modular Avionics,
(IMA) which required applications to be processed on single microprocessor module; so ARINC 653 was born.}

ARINC 653 is a software specification that car run multiple applications on the same hardware, while providing resources in the context of an IMA system.

ARINC 653 is said to “bring a new quality in real-time systems development“,
this is achieved through its API (APEX) which interfaces between the OS and the software applications.
The APEX manages partitions, memory allocation and error handling.
Applications are partitioned separately into those partitions with their own memory and scheduled processing time slots.
These partitions can not interfere with each other and can not take resources from each other.
If an error in a partition occurs, it will only crash itself; the remaining partitions will not be affected.

The idea is to bring such an operating system in to the wider embedded world.
To do that in a one semester project, the focus will be on understanding the standard,
working with embedded systems and implementing the parts of the operating system,
essential for running simple partitions on a single hardware unit.
\todo[inline]{Correct and expand on the introduction to what the standard is about.}

\subsection{All the details}


\section{Hardware}
\subsection{The rise of embedded devices}
\todo{The first paragraph reads a bit too hasty. Should be written a bit longer}
In the world of embedded computing the computing platforms varies a lot more,
compared to the world of workstations and desktopcomputers.
Traditionally embedded computers are used for small specialized static tasks and
not required to run an operatingsystem with multiple processors.
For this reason many embedded systems are arranged as a SoC,
with limitied amounts of ram, flash and a low cpu clock,
not capable of supporting large computer screens and high speed networking.

Improved technology however, have brought powerful new CPUs of architectures used in the embedded world
such as MIPS and ARM.
The latter now being used in all levels of computing, from ultra low power single purpose embedded systems,
to workstations and supercomputer installations with graphics and highspeed networking requirements.
For more powerful computers this might represent both a shift from the widely used CISC architecture
to the smaller RISC architecture and a new focus on power efficiency over increased CPU speed.

For embedded devices this change has meant more capable systems
and an increasingly narrowing gap between embedded and workstation computing.
Capable low power devices and cheap manufacturing of this new generation of RISC based computing systems,
has made it possible to build computers to solve problems on all scales using the same, or simular, platforms.
This has brought a surge of development for the middle layer of computing,
a little more capable thán traditional embedded devices and a little less capable than the typical workstation.

Developing a system to operate in an embedded environment with real time requirements
and low overhead peripheral interaction,
yet with the flexibility and multitasking capabilities of a modern operating system,
puts some lower limits on the capabilities of the hardware.

\subsection{The lower bound on systems}
Overall these requirements must to be met by the hardware:
\begin{itemize}
	\item Be fast enough to run multiple applications and the operating system below
	\item Have sufficient main memory to support the stacks for all running applications and the operating system
	\item Must come with either an MMU or MPU to provide protected regions in memmory for serperate contexts
\end{itemize}

The first two requirements depends almost entirely on the tasks being solved.
The more applications being run and the more resource demanding they are,
the more CPU time and memory they are going to utilize.
The requirement to provide a MMU or MPU to safely segregate blocks of memmory between applications and the operating system,
puts a lower bound on the type of hardware required.

\todo{Give a more detailed explanation on the difference between an MPU and MMU}
More capable systems may provide a MPU for basic memmory protection
The smallest embedded systems typically do not provide any memmory protection,
only being required to run single purpose applications where all drivers and interfacing
are compiled and executed as a single binary image.
Ever more powerfull systems may come with a MMU,
providing a uniform picture of the memory-space from the viewpoint of every application.

\subsection{Some of the options out there}
The best way to decide on a hardware platform to work on is to explore some of the options out there.
Being a single-semester project with varying degree of experience,
implementing an operating system from scratch;
the hardware should not be in the complex end of what is available.

Below is a brief list of hardware platforms listed in no particular order.
Every unit has been selected on the premise of possessing either an MMU or MPU for the space partitioning feature in the OS.
Also the candidates varies to some degree in capability and complexity,
but more than anything; it was just what we could find with a few Google searches.
\todo{Specify the search terms we used to find them.}

\subsection{Settleing on a platform}


\section{Software}
\subsection{Overview}

An ARINC 653 compliant platform has some strict requirements. The OS must perform safety critical and time critical tasks that must be separated by space and time. There are accompanying standards to test and verify a system to determine how well it meets its requirements. 

\subsection{Avionics safety standards}
\subsubsection{DO-178}
DO-178 is a set of guidelines to be used for software assurance in safety-critical software in avionics systems. Use of the standard can determine the reliability of the software in an airborne environment. There is a "safety assessment process" and "hazard analysis" to determine the result of failure conditions on the system. The final result of the standard applied to the software system will give the system a "software level" of A to E compliance, where A is the best level to be achieved and E is the worst. A level A compliance mean the system has been verified and validated against catastrophic failures, whereas an E compliance has been verified against trivial failures.
% \subsubsection{DO-254} this standard is the counterpart of DO-178 (seems to do the same thing)
% \subsubsection{DO-297} cant find much on this standard


\subsection{Alternative ARINC platforms}

When considering how to implement ARINC 653 we first explored existing platforms that might meet our needs. There are COTS (Commerical of the Shelf) OSs, and there are open source OSs. An existing platform would save us time to get an ARINC platform running, but with varying levels of limitations between the COTS and open source. 

\subsubsection{open source OS platforms}

An existing open source ARINC OS could have potentially worked since its limitations would be minimal compared to a COTS solution. %Saving time and money to get an ARINC platform running

\subsubsection{ARLX}
% http://dornerworks.com/portfolio/arlx-arinc-653-real-time-linux-xen
% http://genesysideation.com/news/dornerworks-releases-prototype-arinc-653-hypervisor-to-open-source-community/

ARLX is a prototype hypervisor for ARINC 653 which is an extension to the Xen hypervisor. Linux is used to to partition the system. It has been developed in accordance with DO-178.

% \subsubsection{LithOS} http://www.fentiss.com/en/products/arinc653.html

\subsubsection{COTS OS platforms}

There are several commercial OSs that support ARINC: VxWorks, PikeOS, LynksOS, WindowsCE, Linux RTAI and CsLEOS OS to name the main ones.
%"eliminates the risk of creating and certifying an operating system for new projects"

\subsubsection{VxWorks}
% important reference1 http://www.windriver.com/products/product-overviews/PO_VxWorks653_Platform_0210.pdf
% important reference2 https://windriver.com/products/product-overviews/2691-VxWorks-Product-Overview.pdf
Called VxWorks 653, is it marketed as "The Industry's Leading Operating System for Embedded devices". It contains an ARINC 653 compliant platform and compliance with DO-178C to level A, DO-254 and DO-297. Capable of full virtualization on a shared multi-core processor.



\section{}

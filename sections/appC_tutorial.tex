\chapter{Compile Tutorial}
\label{app:tutorial}
This tutorial is gives a quick step-by-step introduction on how to compile the open653.
It assumes that the reader already have the code and is pursuing the task using a Linux environment
with the following tools already installed:

\begin{itemize}
	\item arm-none-eabi-gcc
	\item make
	\item cmake
	\item openocd
\end{itemize}

The steps to compiling the OS with partitions and
flashing it to the chip are as follows:

\begin{enumerate}
	\item Locate the root folder of the project
	\item Make a new folder to contain the build files (eg. mkdir build)
	\item From within the build fonder run the command `cmake ..'.
		This reads the CMakeLists.txt files and builds make files
	\item Run `make' to build the project or use `make help' to get a list of available targets
	\item If hardware is connected the computer the command `make OS\_writeflash'
		can be used to compile the code and flash the hardware.
\end{enumerate}

The hardware is reset after being (re-)programmed.

\chapter{Conclusion}\label{ch:conclusion}

In this chapter the system developed in this project is reviewed and conclued if the aproach to the problem
 statement from chapter ... on page ... was successful and are the possible uses in the future.\\

\section{Discussion}
The main concerns presented by the forums communities were the security levels of the implementation of this 
OS. The two main topics were how to identify an error and how to track it to the origin and how the system
 will react to different levels of failure, from small to fatal errors. If the entire system would reboot,
  only restart the partition or block it.\\
 Other problems suggested were related to the interrupts, when context switching if the next partition would 
 also have acess to all them, and how the control is made to prevent any errors related.\\
The idea of an OpenSource implementation like this was well accepted due to the lack of examples avaiable. 
It was given an example of an implementation in the USA, of a similar system and that is still used for
communication infrastructures.
https://www.embeddedrelated.com/thread/1326/
\todo[inline, color=green]{Include the refrences and the sites or posts maybe}

\section{Perspective}

As this implementation can save the amount of the hardware needed for multiple processes, saving than
 companies' money, it can have a wide usage in the future. It can be used in car's industries, telecommunication, distributed systems and other embedded environments. 
 As the OS has ports for communication it allows the creation of a web of processes's communicating with 
 each other using the same protocols and avoiding hardware or software problems.
\\
\section{The actual conclusion}

\todo[inline, color=green] {could be a different chapter if it fits our needs}
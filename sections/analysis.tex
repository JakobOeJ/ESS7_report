%! tex root = ../master.tex

\chapter{Analysis}\label{chap:analysis}
This chapter includes information about
the standard, the hardware going to be used as well as the programming
The \arinc{} requirements are analysed, while looking at existing projects
and other relevant standards in the field of avionics. Based on this,
decisions can be made regarding the following stages of \OSname\ \textquotesingle s
development, and narrowing the project\textquotesingle s
scope.

\section{Real-time systems}
A real-time system is handling data as it comes in.
Besides the usual computations needed to be done on data,
the worst case execution time of the computation is also important.
%Besides the usual computations needed to be done,
%the time until a result is generated is also important.
This means that the applications are scheduled in a manner
that would allow them to respond predictably.
Real-time systems are developed in order to answer
requirements of different fields of work
where time is critical: military, automotive or avionics.

\section{Integrated Modular Avionics}
In the avionic industry, the devices and subsystems used for
controlling the peripherals have to be reliable and operate using
fixed time constraints. Integrated Modular Avionics is a term used
to describe a distributed real-time computer network,
deployed in the avionics industry\cite{ima_description}.
Modularity, keeping track of the time and criticality levels are
the main characteristics of such a network. IMA also aims to centralise data 
processing by routing sensor data to only a few compute nodes, rather than
processing data at the sensor. This implies that the applications running would 
share the system\textquotesingle s resources following a set of well defined 
rules.

The main benefits of using the IMA concept are: aircraft weight reduction\cite{boeing_weight_reduction}
and lower maintenance costs\cite{arinc_scarlett}.
There are certain guidelines how such a system
should be designed and implemented, as well as standards that specify
how space and time should be partitioned.

\section{ARINC 653}
\label{anal:arinc}
\arinc{} is a software specification that allows a system to host
multiple avionics applications on the same hardware, while being able
to execute them independently. This is achieved by partitioning the
system\cite{arinc_page_2}.

\subsection{Overview}
\arinc{} is said to \textit{bring a new quality in real-time systems
development}\cite{arinc_scarlett}.
This is achieved through its API (APEX), the interface between
the OS and the software applications.
Applications are residing in partitions,
which have their own memory and scheduled processing time slots.
These partitions can not interfere with each other and can
not take resources from each other.
If an error in a partition occurs, it will only crash itself;
the remaining partitions will not be affected.

The following is a brief description of what the standard specifies.
\\\\
\textbf{Modules}
refer to the aviation concept of IMA, where a module
can be seen as a piece of hardware, controlling other subsystems.
A module can contain one or more processors. However, this does not fit 
well with the concept of partitions,
which have to run on a single processor\cite{arinc_page_11}.
A processor used for \arinc{} applications should have sufficient processing
power and access to the required inputs and outputs,
as well as memory and time resources.
Besides this, the processor should be able to isolate a partition from
the others, if it fails\cite{arinc_page_12}.
\\\\
\textbf{Partitions}
are program units designed to obey space and time limitations
set by the developer. The way partitions are being run by the processor,
is in a static, cyclic manner meaning that they do not have any priorities.
One partition should only have writing rights to its own memory space\cite{arinc_page_13}.
The resources used by the partitions are specified at build time\cite{arinc_page_14}.
The standard differentiates between partitions that run applications and system partitions. The latter are optional, and can be used to contain
services not provided in the APEX\cite{arinc_scarlett}.
\\\\
\textbf{Processes}
are the parts of the program running on partitions. One partition
may contain one or more processes. These may operate concurrently,
and may operate in a periodic or aperiodic manner.
A process\textquotesingle s
characteristics are: data and stack areas, program counter, stack pointer,
priority and deadline. This makes them resemble threads in the POSIX\cite{posix} process model.
Certain attributes are statically defined, and cannot
be changed during operation\cite{arinc_page_19}.
Process management is done with the use of APEX calls.
Processes can be in different states: dormant, waiting, running, ready.
Their scheduling is done in such a way that the process in the ready state
with the highest priority, is always executing when the partition is active.
This entails processes being preempted at any time, by the process with a
higher current priority\cite{arinc_page_20}.
\\\\
\textbf{Time management}
is very important in real-time systems.
Time keeping is needed for partition and process scheduling,
for delays, deadlines as well as for communication time-outs.
Partitions receive time-slots during system configuration, and
processes receive time capacity\footnote{Time given to them to satisfy
their process requirements. This is an absolute duration, not execution time} when they are started.
The difference between the two brings the need for having extra APEX calls
to control and monitor how processes are behaving\cite{arinc_page_25}.
\\\\
\textbf{Memory spaces}
are defined during the system configuration. These can not
be changed later on.
\\\\
\textbf{Interpartition communication}
is an important aspect of \arinc{}. It provides a framework for processes across different partitions
to communicate with each other. This applies to partitions placed
on the same module, on separate core modules, as well as communicating with non-\arinc{} modules.
This communication is based on message passing. Channels are used as
logical links between partitions. Partitions can access channels through
ports, which allow monodirectional access to a channel\cite{arinc_page_27}.
These ports are defined during system configuration and are tied to
both a certain channel and a certain partition.
\\\\
%------------------------------------------------------------------
\iffalse
The \arinc\ specification defines a way for different partitions to communicate with one another.
This communication is based on message passing. To transfer messages between partitions the \arinc\ standard
defines the concepts of channels and ports.

\paragraph{Channels} are logical links spanning partitions. If necessary, 
a channel might logically span different modules. 
\todo{Has modules been defined yet?}
The way for partitions to access a channel, is via ports.
Channels are defined by the system integrator at configuration time.
Channels should be transparent to the application programmer.

\paragraph{Ports} allows monodirectional access to a channel;
a port can either be configured send or receive messages by a partition.
When a message is written to a port, it will remain in a buffer, until the operating system transmits the message to the destination.\\

Two types of ports exist, sampling ports and queuing ports. 
\subparagraph{Sampling ports} employ no form of message queuing; if a partition tries to write more than one message to a sampling port
before they are transmitted through the channel to the destination port, the prior message will be overwritten. Similarly, once a message arrives at the destination port,
it will override any prior messages, regardless of whether it had been read or not. These ports may be used in situations where only the most recent message is important.
\subparagraph{Queuing ports} queue messages both at the sender and at the receiver. Ports in queuing mode may employ a queue buffer to keep more than one message at a time.
When processes send messages they accumulate in the source port queue. At a behest of the operating system, messages in the source port queue will be transmitted to the destination port queue.
When a process reads a message from the destination port, that message will be cleared from the port queue. If the destination queue is filled, no more messages will be allowed
in to that queue until it is emptied. If the destination queue is filled, messages may start to accumulate in the source port queue. If a process tries to send a message through a
source port with a full queue, the process may block until it is cleared, or receive an error.
Process waiting for a queue to have space (source) or to have content (destination), can either wait in a purely FIFO order or a priority based FIFO order, where processes with high priority
will be moved up in the FIFO queue.\\

Messages sent through queuing channels should always be sent in a FIFO manner.
Ports are defined at solely configuration time and are tied to both a certain channel and a certain partition.
It is therefore impossible for application code to define new ports at runtime. 
For a process to access a port, an object representation must be created. This happens through an APEX system call.
The location in memory of port buffers is not specified in the \arinc\ specification.
\fi
%------------------------------------------------------------------
\textbf{Intrapartition communication}
allows processes inside a partition to communicate to each other. This way
the communication is done more efficiently than having to use a
global message passing system.
This can be done by implementing buffers and blackboards, or semaphores and
events\cite{arinc_page_36}.
\\\\
\textbf{Health Monitor}
Has the purpose of monitoring, reporting and isolating hardware or
software faults and failures. Its responsibilities are divided into
the following levels: process, partition and module. The health monitor has
error response mechanisms to take action in case typical
error occurs at any of these levels
\cite{arinc_page_40}.
\\\\
The standard allows portability and reusability of applications. However,
these applications first have to be integrated in the system. This is done
by giving the partitions the access and resources needed,
in the \textbf{system configuration} phase, in order to
satisfy the applications\textquotesingle{} requirements\cite{arinc_page_42}.

\subsection{Decision}
The idea is to bring such an operating system based on \arinc{}
in to the wider embedded world.
The focus will now change to understanding if, and how other projects
have approached this idea.

\section{Other existing standards and ARINC implementations}
While \arinc{} defines the architecture of a system for safety-critical
avionics applications, there are other ways to categorize the reliability of such a system.
Usually this is measured by looking at the number
of failures occurring in time frame.
The following standard attempts to grade a systems reliability.

\subsection{DO-178B}
DO-178B\cite{do178b_wiki} is a set of guidelines to be used for
software assurance in safety-critical avionics systems.
Use of the standard can determine the reliability of the software in an airborne environment.
There is a \textit{safety assessment process} and \textit{hazard analysis} to determine the result of failures in the system.
Applications are graded with a \textit{software level} of A to E,
where A is the best level to be achieved and E is the worst.
A level A compliance means the system has been verified and validated against catastrophic failures,
whereas an E compliance has been verified against trivial failures.
Its successor, DO-178C\cite{do178c_wiki} was released in January 2012.

\subsubsection{Certification of \arinc{}}
The concept of having partitions
to host avionics applications, is already seen as a step
forward in terms of safety.
Applying the DO-178B standard on an ARINC implementation would certify
that this is ready to be deployed in the avionics field with a grade.
In fact this is what similar projects are doing.

\subsection{Alternative \arinc{} platforms}
There are a number of \arinc{} implementations available, which is not
surprising, having understood the benefits of IMA.
These are divided open source OSs and COTS\footnote{Commercial of the Shelf} OSs.
An existing platform could be used as a starting point for
the implementation of \OSname{}.

\textbf{ARLX}\cite{arlx}
is an open-source platform available. It is a bare-metal hypervisor 
for ARINC 653 based on the Xen hypervisor, with a Linux partition as the main
virtual machine (in Xen terminoligy, domain 0).
It has been developed in accordance with DO-178C.

\textbf{Commercial platforms}
such as VxWorks 653 (DO-178C certified)\cite{vxworks},
PikeOS (DO-178 compliant)\cite{pikeos},
LynxOS-178 (DO-178B certified)\cite{linxos}
support \arinc{}.

As an example, \textbf{VxWorks 653}
is marketed as \textit{The Industry's Leading Operating System for Embedded devices}\cite{vxworks}.
It contains an \arinc{} compliant platform and is compliant with DO-178C to level A.

As mentioned, there is no surprise that the existing
\arinc{} implementations are all DO-178 certified. While it is
important for them to work properly, the compliance to \arinc{} and the
certification of a high level in DO-178 shows that they are safe to use
in critical applications. %Apart from this, it makes the developer\textquotesingle s
%life easier as well as the life of the administrative department.

%\subsubsection{The developer's point of view}
%It is easy for a developer to create and maintain a framework that has
%strict time and space partitioning.
%When new applications need to be added to the system, all the developer
%needs to know is the level the applications have been DO-178 certified to.
%This way he could then provide the proper resources for the application
%to meet its goal.
%\iffalse
%("Assuming each application is certified for DO-178B/C at a given level (A through E), safety integrity of the overall avionics systems can be achieved since data integrity can be guaranteed and a sufficient time budget can be allocated to each application so that it can run in due time. ")
%\fi
%\subsubsection{The administrative point of view}
%It is easier to get an \arinc{} framework DO-178 certified, and build on top
%of that. The partitions would make it easier to divide the application
%code into multiple sections, that could be certified independently.
%Updating an application would be a matter of rewriting the code and
%getting just this part recertified, instead of the whole system.
%\iffalse
%("The ability to achieve incremental certification is a direct benefit of the ARINC 653 partitioning mechanisms since certification can be achieved first for each application and then for the overall systems. Moreover, changing an application does not require recertification of the other applications.")
%\fi

\subsection{Decision}
The research on the existing \arinc{} platforms concluded with at
least four options. The commercial OSs are excluded from the start since
they are not in the project\textquotesingle s scope. The only remaining
variant, the ARLX has to be excluded as well, since it is a hypervisor,
and does not fit the scope of the Masters programme.
There has been gained in-depth knowledge about certification and other standards, but for the rest of this document, only the \arinc{}
specification and design consideration will be used.
The fact that the group does not work on top of an existing implementation
increases the degree of freedom for choosing a hardware platform to fit
the scope of the project and of the education programme.

%------------------------------------------------------------------
\iffalse
\section{Some of the options out there}
\todo[inline]{this was rewritten above, with all the subsections}
As seen in the previous section \arinc{} compliant platform has some strict requirements.
The OS must perform safety critical and time critical tasks that must be separated by space and time.
There are accompanying standards to test and verify a system to determine how well it meets its requirements.

\subsection{Avionics safety standards}

\subsubsection{DO-178}
DO-178 is a set of guidelines to be used for software assurance in safety-critical software in avionics systems.
Use of the standard can determine the reliability of the software in an airborne environment.
There is a ``safety assessment process'' and ``hazard analysis'' to determine the result of failure conditions on the system.
The final result of the standard applied to the software system will give the system a ``software level'' of A to E compliance,
where A is the best level to be achieved and E is the worst.
A level A compliance mean the system has been verified and validated against catastrophic failures,
whereas an E compliance has been verified against trivial failures.
% \subsubsection{DO-254} this standard is the counterpart of DO-178 (seems to do the same thing)
% \subsubsection{DO-297} cant find much on this standard

\subsection{Alternative ARINC platforms}
When considering how to implement ARINC 653 we first explored existing platforms that might meet our needs.
There are COTS (Commercial of the Shelf) OSs, and there are open source OSs.
An existing platform would save us time to get an ARINC platform running,
but with varying levels of limitations between the COTS and open source.

\subsubsection{Open source OS platforms}
An existing open source ARINC OS could have potentially worked since its limitations would be minimal
compared to a COTS solution. %Saving time and money to get an ARINC platform running

\subsubsection{ARLX}
% http://dornerworks.com/portfolio/arlx-arinc-653-real-time-linux-xen
% http://genesysideation.com/news/dornerworks-releases-prototype-arinc-653-hypervisor-to-open-source-community/
ARLX is a prototype hypervisor for ARINC 653 which is an extension to the Xen hypervisor.
Linux is used to partition the system.
It has been developed in accordance with DO-178.
% \subsubsection{LithOS} http://www.fentiss.com/en/products/arinc653.html

\subsubsection{COTS OS platforms}
There are several commercial OSs that support ARINC:
VxWorks, PikeOS, LynksOS, WindowsCE, Linux RTAI and CsLEOS OS to name the main ones.
%"eliminates the risk of creating and certifying an operating system for new projects"

\subsubsection{VxWorks}
% important reference1 http://www.windriver.com/products/product-overviews/PO_VxWorks653_Platform_0210.pdf
% important reference2 https://windriver.com/products/product-overviews/2691-VxWorks-Product-Overview.pdf
Called VxWorks 653, is it marketed as ``The Industry's Leading Operating System for Embedded devices''.
It contains an ARINC 653 compliant platform and compliance with DO-178C to level A, DO-254 and DO-297.
Capable of full virtualization on a shared multi-core processor.

\subsection{Decision}
\todo[inline,color=green] {why we did not use any of the Arinc implementations mentioned above?}
Since we decide not to use any of the existing implementations...
the degree of freedom for choosing the hardware is increased.
\fi
%------------------------------------------------------------------

\section{Hardware}
Having the \arinc{} specification basic prerequisites in mind, the
different options for hardware are investigated in this section.

\subsection{The rise of embedded devices}
In the world of embedded computing, the computing platforms vary a lot,
compared to the world of workstations and desktop computers.
Traditionally, embedded computers are used for small specialised static tasks and
not required to run an operating system with multiple processes.
For this reason many embedded systems are arranged as a SoC,
with limited amounts of ram, flash and a low CPU clock,
not capable of supporting large computer screens and high speed networking.
\\\\
Improved technology however, have brought powerful new CPUs of architectures used in the embedded world
such as MIPS and ARM.
The latter now being used in increasingly many levels of computing, from low power single purpose embedded systems,
to workstations and supercomputer installations with graphics and high speed networking requirements.
For more powerful computers this might represent both a shift from the widely used CISC architecture
to the smaller RISC architecture and a new focus on power efficiency.
\\\\
For embedded devices this change has meant more capable systems
and an increasingly narrowing gap between embedded and workstation computing\cite{raspberry_specs}\cite{windows10_arm}.
Capable low power devices and cheap manufacturing of this new generation of RISC based computing systems,
has made it possible to build computers to solve problems on all scales using the same, or similar, platforms.
This has brought a surge of development for the middle layer of computing,
a little more capable than traditional embedded devices and a little less capable than the typical workstation.
\\\\
Developing a system to operate in an embedded environment with real time requirements
and low overhead peripheral interaction,
yet with the flexibility and multitasking capabilities of a modern operating system,
puts some lower limits on the capabilities of the hardware.

\subsection{The lower bound on systems}
Overall these requirements must be met by the hardware:
\begin{itemize}
	\item Be fast enough to run multiple applications and the operating system
	\item Have sufficient main memory to support the stacks for all running applications and the operating system
	\item Must come with either an MMU or MPU to provide protected regions in memory for separate contexts
\end{itemize}
\noindent
The first two requirements depend almost entirely on the tasks being solved.
The more applications being run and the more resource demanding they are,
the more CPU time and memory they are going to utilize.
The requirement to provide an MMU or MPU to safely segregate blocks of memory between applications and the operating system,
puts a lower bound on the type of hardware required.
\\\\
More capable systems may provide an MPU for basic memory protection.
The smallest embedded systems typically do not provide any memory protection.
Only being required to run single purpose applications where all drivers and interfaces
are compiled and executed as a single binary, hence they can expect no foreign element disrupting their execution.
More powerful systems may come with a MMU,
providing a memory sandbox of each application.

\subsection{Some of the options out there}
The best way to decide on a hardware platform to work on is to explore some of the options out there.
Implementing an operating system from scratch,
the hardware should not be in the complex to work with.
\\\\
In appendix \ref{app:embedded_devices} is a brief list of hardware platforms listed in no particular order.
Every unit has been selected on the premise of possessing either an MMU or MPU for the space partitioning feature in the OS,
also the candidates vary to some degree in capability and complexity.
\\\\
These are the evaluated hardware platforms:\\

\textbf{Creator Ci40} is a platform meant for IoT development.
It comes with easily available ports and connections for embedded development
and networking over protocols like ethernet and bluetooth.
It is built around the cXT200 chip which is based on a JZ4780 550MHz dual-core MIPS CPU \cite{creator_ci40_specs}.
It comes with an MMU, is capable of running Linux
and is generally more powerful than many low power embedded platforms.

\textbf{MINI-M4 for STM32} is an ARM Cortex-M4 based development board containing a STM32F415RG microcontroller.
It provides a simple pin layout and fits well into a breadboard for connecting external devices
and comes with a single USB port to deliver power and to be programmed through, with an installed bootloader.
The STM32 chip comes with a MPU and a single core CPU capable of running 168 MHz,
but limited to 120 MHz using the on-board bootloader \cite{MINI-M4_stm32_specs}.
Available is a downloadable hardware abstraction layer (HAL) library from chip manufacturers website,
which comes with a number of implementation examples and documentation \cite{HAL_library}.

\textbf{MINI-M4 for Tiva} is similar to the MINI-M4 for STM32 above,
based on the same ARM Cortex-M4 architecture,
but using the TM4C123GH6PM chip from Texas Instruments instead.
It has a pin layout similar to the MINI-M4 for STM32,
a USB connection for power and programming and comes with an MPU,
but only runs at 80 MHz \cite{MINI-M4_tiva_specs}.

\textbf{Raspberry Pi 3} is
based on a 1.2GHz 64-bit quad-core ARMv8 CPU which
makes it available for both embedded use and common desktop computing tasks \cite{raspberry_specs}.
The Raspberry Pi lacks any comprehensive documentation on its register layout,
and being a higher end device compared to MINI-M4 and Creator Ci40,
it's presumably more complex to work with.

\subsection{Settling on a platform}\label{section:settleing_platform}
Obviously a list longer than the one seen above can be compiled.
Although the hardware used is not the primary concern to the project
and as such the goal is to show examples of known platforms,
available for purchase in Denmark,
that all hit within the range of what can be considered embedded devices.
\\\\
The two MINI-M4 devices represent a lower, but capable end of the performance spectrum,
while the Raspberry Pi and the Creator Ci40 are more capable.
All four devices have the necessary features to run a simple low power OS,
but building the OS gives rise practical concerns to the complexity and ease of setup.
\\\\
Focusing on core OS features, only a few basic peripherals
like timers, MPU/MMU, UART and basic IO are necessary for the project.
The ARINC 653 does not provide a solution to scaling the operating system
or containers across multiple cores,
hence more capable hardware is likely to see a low utilisation.
Since the project is focused on the implementation of an OS
and less so the partitions running on it.
\\\\
The MINI-M4 modules are more attractive choices from the utilisation standpoint.
A comprehensive datasheet is available for both of them,
making it easier to program them from scratch.
Some experience with the STM32 is present within the project group
on how to setup a Linux based tool-chain. The manufacturer of the STM32
provides sample code implementations as well as hardware abstraction
layer libraries, which would give a head start on the project.

\subsection{Decision}
This leads to the decision of using Mini-M4 for STM32 as the hardware
for \OSname{}.

\subsection{Communication with the board}
In order to get started with the development of the OS, there is a need for
communication with the board. Due to previous experience with
this type of hardware, the following components have been ordered:

\begin{itemize}[noitemsep]
	\item ST-LINK/V2 in-circuit debugger/programmer for STM8 and STM32
	\cite{st_link} - to give access to the JTAG port
	\item FTDI TTL-232R-3V3 cable \cite{ttl_usb} - to create a communication
	port
\end{itemize}

\subsection{Helper libraries}
When starting to write software, one needs to decide how to make this
process easy and transparent. This is very dependent of the
project\textquotesingle s requirements such as time frame, performance,
security and reusability. One possibility would be to work on the
"bare metal", and
use the chip\textquotesingle s reference guide\cite{reference_manual}, and maybe
part of the ARM architecture reference manual\cite{arm_architecture}.

Another way could be to use some libraries that define the I/O and work
on a4 higher level. ARM provides libraries for their architecture
as well as the vendors providing libraries and examples on how to use their
hardware.

\subsection{The legacy from the AAU Racing project}
This project has some roots in another project that has been going on
for some years at Aalborg University \cite{aauracing}.
The choice of hardware was
slightly influenced by this, as well as the tools and libraries used
throughout the project. The big advantage of this, is that by having the
experience and knowledge the project could get a head start.
\todo{Maybe Jakob has something to add}

\section{Programming language choice}
The languages Ada, C and C++ are taken in consideration as an embedded language.
\\\\
Ada was created during the 70\textquotesingle{}s by the Defense Department of the USA.
It was created to reduce the number of
languages used between departments.
It is a high level language with real-time capabilities.
\\\\
C is a simple low level language especially popular in embedded systems.
Like Ada it was created in the 70\textquotesingle{}s and
among the most popular languages used today.
C is easy to use in projects favoring memory and instruction control,
as is favored with ARINC 653,
where resources are allocated statically.
\\\\
C++ is a multi paradigm language described by some as ``C with classes''.
Created in the late 70\textquotesingle{}s and first standardised in the 98, it
is a language with both high-level object oriented features as well as low-level
memory management. It is mostly compatible with C. C++ contain many extra
features compared to C, such as classes. These can both benefit and complicate
code.
\\\\
For the passing of the schema, a higher level language can be used
since it is running on the computer used in development,
hence there is little concern for optimisation and memory usage.
The schema will be translated into a usable format to the embedded system.
Once translated it will be compiled and uploaded to the
embedded system.
\\\\
Python is an easy high level language to use.
There is very little boiler plate code involved,
it is not typeset, nor does it need to compile
before it is executed.
It has many libraries to interact with structured
data types such as XML and JSON,
which makes it a well suited option for developing the schema parser.

\subsection{Decision}
Python is chosen as the language to parse the schema because of its simplicity
and because it is cross-platform.
C is chosen for embedded development because of its simplicity and
the level of experience within the group, with the language.\\
The choice of these languages also mean that the resulting code can be compiled
from all three major PC operating systems.

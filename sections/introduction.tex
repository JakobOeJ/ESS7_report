\chapter{Introduction}\label{ch:introduction} 

Embedded devices are becoming more powerful and less expensive by the day. The large majority of
microprocessors produced end up as components of embedded devices, ranging from mobile phones
to traffic lights, from digital watches to medical apparatus. These devices are tailored for 
performing specific tasks, thus requiring the developers' skill and commitment to create reliable
systems. The diverse applications of these systems divide them into different categories, in regards to
how dependable they have to be. A wristwatch skipping a second has no severe consequences, while the
failure of a surgical robot has to be avoided at all costs. These requirements led to the creation 
of technical standards to be followed when developing such systems. 
\\\\
Some might think that these regulations are unfairly formulated and distributed, and in most
of the cases they might be right. For example, if we compare the existing standards for the aeronautical industry with the ones in the automotive field, there are quite big discrepancies. The standards 
in use in the automotive industry are applying certain rules from the avionics regulations, 
but only to a certain extent
\cite {can_cars_fly}.
However, these two fields seem to be converging in terms of safety requirements as the automotive industry develops.
\\
This gives rise to this project exploring how these avionics safety critical systems could be applied to other industries.
There is potential for reducing weight of products with a single computing device,
facilitating more safety proceedures
or increasing portability of applications across generations of products.
\\
This project\textquotesingle s cornerstone is developing an Operating System, that facilitates runing safety critical applications compliant with the 653 standard. 
The two main areas of interest are:
\\
\begin{itemize}
\item The ARINC 653 Standard, that specifies an IMA approach to run applications
\item The physical platform that would contain the system and serve as a research launchpad
\end{itemize}
The \nameref{chap:analysis} chapter provides more details about the software specification that needs to be fulfilled,
as well as the different available hardware options.
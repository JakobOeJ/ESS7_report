\chapter{Analysis}
\todo[inline,color=green]{Intro for te analysis.}

\section{Realtime systems}

\section{Integrated Modular Avionics}
This is about the IMA.

\section{ARINC 653}
\subsection{Overview}
ARINC 653 is a standard that was originally designed and created for the avionics industry.
It has been created to work within Integrated Modular Avionics,
(IMA) which required applications to be processed on single microprocessor module; so ARINC 653 was born.
ARINC 653 is said to “bring a new quality in real-time systems development“,
this is achieved through its API (APEX) which interfaces between the OS and the software applications.
The APEX manages partitions, memory allocation and error handling.
Applications are partitioned separately into those partitions with their own memory and scheduled processing time slots.
These partitions can not interfere with each other and can not take resources from each other.
If an error in a partition occurs, it will only crash itself; the remaining partitions will not be affected.

The idea is to bring such an operating system in to the wider embedded world.
To do that in a one semester project, the focus will be on understanding the standard,
working with embedded systems and implementing the parts of the operating system,
essential for running simple partitions on a single hardware unit.
\todo[inline]{Correct and expand on the introduction to what the standard is about.}

\subsection{All the details}

\section{Hardware}
We know everything about the hardware.
\missingfigure{We need a figure right here!}
\todo[inline,color=green]{Hardware analysis.}

\section{}

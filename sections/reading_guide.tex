\subsection*{Reading Guide}
\label{sub:reading_guide}

In this report, references are written using the Harvard method [last name(author or company),year(if available)].
Sources are listed alphabetically in the bibliography.
Figures, Tables, Equations, and Listings (code snippets) are numbered after the chapter they are in,
for example, the first figure in chapter two is called 2.1, the next 2.2 etc.
All illustrations and figures are made by the group members unless otherwise stated.
All data sheets and code can be found on the zip archive electronically uploaded with the report.

A prerequisite for reading this report is basic knowledge in the 
computer science field.

\subsubsection*{Glossary}
\begin{longtable}{l p{12cm}}
\textbf{APEX}		& \qquad \qquad Application/Executive\\
\textbf{CCM data RAM}	& \qquad \qquad Core coupled memory data RAM\\
\textbf{COM}		& \qquad \qquad Communication port\\
\textbf{COTS}		& \qquad \qquad Commercial off the shelf\\
\textbf{CPU}		& \qquad \qquad Central Processing Unit\\
\textbf{CMSIS}		& \qquad \qquad Cortex Microcontroller Software Interface Standard\\
\textbf{EWI}		& \qquad \qquad Early Wakeup Interrupt\\
\textbf{GB}			& \qquad \qquad Gigabyte\\
\textbf{GPIO}		& \qquad \qquad General-purpose input/output\\
\textbf{HAL}		& \qquad \qquad Hardware Abstraction Layer\\
\textbf{IDE}		& \qquad \qquad Integrated development environment\\
\textbf{IMA}		& \qquad \qquad Integrated Modular Avionics\\
\textbf{IoT}		& \qquad \qquad Internet of Things\\
\textbf{ISR}		& \qquad \qquad Interrupt Service Routine\\
\textbf{JTAG}		& \qquad \qquad Joint Test Action Group\\
\textbf{KB}			& \qquad \qquad Kilobyte\\
\textbf{MB}			& \qquad \qquad Megabyte\\
\textbf{MMU}		& \qquad \qquad Memory Management Unit\\
\textbf{MPU}		& \qquad \qquad Memory Protection Unit\\
\textbf{NVIC}		& \qquad \qquad Nested Vectored Interrupt Controller\\
\textbf{OS}			& \qquad \qquad Operating System\\
\textbf{RISC}		& \qquad \qquad Reduced instruction set computing\\
\textbf{RTC}		& \qquad \qquad Real-time clock\\
\textbf{SRAM}		& \qquad \qquad Static random-access memory\\
\textbf{UART}		& \qquad \qquad Universal Asynchronous Receiver/Transmitter\\
\textbf{USB}		& \qquad \qquad Universal Serial Bus\\
					& \\
\textbf{.txt}		& \qquad \qquad Text file\\

\end{longtable}

%! tex root = ../master.tex

\newcommand{\excreturn}{EXC\_RETURN}

\chapter{System Implementation}

\todo[inline, color=green]
{INTRO}


\section{Hardware}

\section{Drivers}

\subsection{UART}

The UART sets the speed of the transmission and reception of the information between the STM32 and the outside. The communication is made by one character at a time. The transmission is activated by the interrupts and it can happen both ways simultaneously, or one way at a time. An UART has a clock generator, that we set for 16Mz, input and output registers, transmit/receive control, read/write control logic, and transmit/receiver buffers.
It is also possible to communicate with the UART using a screen, getting some important output for debugging.

\subsection{Watchdog timer}
% \ http://electronics.stackexchange.com/questions/123080/independent-watchdog-iwdg-or-window-watchdog-wwdg
The STM32 MCU has two watchdog timers ready to reset the program, in case of an error.
The first one, is the independent watchdog timer, based on an independent oscillator (32 kHz).
When initialised, the watchdog timer receives a value, from which it starts counting down, using
the ticks from the crystal. In the meanwhile, the program has to reset this count, by refreshing 
the timer to the same value it used before. If this fails to happen, when the count reaches zero, 
the watchdog timer will react by restarting the board.
\\
The second watchdog timer, is called a window watchdog. In contrast to the independent watchdog, 
one can provide a time interval in which the timer has to be reset. Resetting the timer outside this interval(as well as not resetting it in the mentioned time window) would restart the program. It uses 
the system clock, which means that if this fails, the watchdog won't be able to reset the system.

\subsubsection{Specifics}
The driver for the independent watchdog works by setting a prescaler on the 32kHz crystal to match 
the time requirements of the program(multiples of milliseconds). The watchdog is then initialised 
and started using the HAL library specific functions. \\
When the program is run, the watchdog restarts it after the interval of time set in the initialisation
phase runs out. In order to have the program running as usual, the refresh function has to be called,
 at intervals shorter than the initial time span.

\subsection{Timing function}
This driver is used for precise measuring the execution time of different functions. It uses the DWT
 registers defined in the ARM-M4 Architecture manual. These are responsible for Data Watchpoint 
 and Trace support. By using one of these counters, one could follow the system's clock ticks. 
 At a core frequency of 168 MHz, each tick would take 5.45 nanoseconds.\\
In order to do this, the DWT\textunderscore CONTROL register is set to a value that allows the system's clock to be sent to DWT\textunderscore CYCCNT. After this, the time can easily be tracked by converting the value of DWT\textunderscore CYCCNT.

\section{OS}

\subsection{Scheduler}

\subsubsection{Context Switching}
There are multiple different states the processor can be in, when a context 
switch from one process to another occurs. The state determines whether the
Master Stack Pointer or the Process Stack Pointer was used by the process,
and whether the FPU was used by the process or not. Because of this, the context
switch algorithm must start by determining the processor state in order to know
which registers
to save and where to save them to.\\
The processor state is saved in the Link return register, and it can be one of
the possible six EXC\_RETURN values (see figure \ref{tab:exc-return}).
The registers are saved entirely on the stack. Therefore, it is important for
the software to know which
stack pointer to use when saving the remaining registers. So the very first
thing the interrupt service routine (ISR) used for scheduling must do, is to inspect
the \excreturn\ value in the LR register to determine wether to use the 
Master Stack Pointer or the Process Stack Pointer. The software must then, at 
the address denoted by the stack pointer, save the registers that have yet to be
saved, and update the stack pointer, before saving both the stack pointer and
the \excreturn\ value to the process structure elsewhere in memory. This is all
done in assembly to make sure that there is complete control over which
registers are used, so as to avoid overwriting any registers that have yet to be
saved. This is absolutely necessary, as otherwise the state of register for a
process might be corrupted during a context switch. Additionally, any ISR used
for scheduling must be declared as naked\footnote{A naked function has no
compiler generated prologue and epilogue. This means that the compiler does not
save any registers on the stack before using them, and it does not restore them
before ending a function.}, to avoid the compiler changing registers before the
context can be saved, and to give complete control over which value is pushed to
the PC register at the end of the ISR.\\
After the context has been saved, control is handed over to the scheduler, which
can then decide which process should be running next.\\
Once all the operations the ISR has completed, the context of the process
selected by the scheduler must be restored. From this point, the code must once
again be written in assembly to avoid overwriting a register after it has been 
restored. First, the registers saved by software on the stack must be restored,
then based on the \excreturn\ value the process interrupted with, it can be 
determined to which stack pointer register the new stack pointer must be pushed
to. Lastly, execution must return to the process by pushing the \excreturn value
to the PC register. The \excreturn value pushed to the PC register must be the
same as that read from the LR register when the process was interrupted.


\section{APEX}

\section{Partitions}

\section{Features of the system}
\todo[inline, color=green]{Here we can have a table showing what features
our system has, what features were not implemented, in regards to the
standard}

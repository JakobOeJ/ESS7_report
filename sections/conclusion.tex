\chapter{Conclusion}

\OSname{} works. 
Partitions and scheduling configurations specified 
in the schema are followed through to a working 
product that can be observed physically in the real world. 
The scope defined and implemented demonstrates the 
execution of space and time partitioned applications.
\\
Despite this success, many of the features are not 
complete or contain compliance mistakes. 
Small inconsistencies exist due to lack of foresight
as the project evolved and members collaborated 
in the same domains of the system.
\\\\
Ultimately the goal has been achieved and the 
foundation for future development has been set.
The questions presented in the \nameref{chap:Problem_Statement} can be addressed as follows:
\\\\
\textit{Which features must be developed to make \OSname{} functional and
testable?}\\
The features required to make the system functional and testable are listed in \nameref{design:features} section \ref{design:features}
and the implemented features are listed in \ref{feature_list}.\\

\textit{How can this system be developed without the use of pre-existing
implementations?}\\
\OSname{} is developed as a ground up implementation on top of the MINI-M4.
The HAL library has been used to supplement the development of most drivers and hardware specific functions.\\

\textit{How is this system implemented using C and Python?}\\
Python is used for parsing the XML into C structures which is used for building the source code.
C is used to program \OSname{} and its partitions.\\

\textit{What are the main obstacles when developing such a system?}\\
The biggest challenges faced in the project were:
\begin{itemize}
	\item Understanding the need for using special GCC attributes when constructing special functions for context switching
	\item Establishing a JTAG connection when assembling the hardware
	\item Deciding on an auto-generated data structure
\end{itemize}

As so, the Conclusion answers the main question of this project:\\
\textit{Can an OS implementation based on a subset of the \arinc{} standard be
developed on a Mini-M4 for STM32?}
\clearpage

\section{Discussion}

\section{Perspective}
The main concerns presented by the forums communities were the security levels of the implementation of
this OS. The two main topics were: how to identify an error and how to track it to the origin and how the
system will react to different levels of failure, from small to fatal errors. If the entire system would
reboot, only restart the partition or block it.\\
Other problems suggested were related to the interrupts, when context switching if the next partition
would also have access to all them, and how the control is made to prevent any errors related.\\
The idea of an OpenSource implementation like this was well accepted due to the lack of examples available.
It was given an example of an implementation in the USA, of a similar system and that is still used for
communication infrastructures.
The whole conversation is presented in Appendix \ref{chap:Forum Responses}.

\section{Reflection}
The implementation of \OSname{} as a whole provided sufficient challenges
for each of the group members to develop themselves, along with following
the education\textquotesingle s requirements.
Based on a multitude of decisions taken in the course of the project,
the degree of freedom the team had was rather large,
especially in the early phases.

Due to this, the software development process flowed naturally. 
Also, the nature of the project
allowed the team to cover different areas at the same time, while still
doing progress, without having a strict plan. Seen as a whole, the project
was pursued by applying the principles of agile software development.

One thing to be proud of is that the project never came to a complete halt;
nothing the group set out to do was given up on.

The diversity of the group members has proven to be a challenge due to the different work ethics,
but also beneficial for covering the many aspects of building an operating system.

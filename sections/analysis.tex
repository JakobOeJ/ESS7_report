\chapter{Analysis}
\todo[inline,color=green]{Intro for te analysis.}

\section{Realtime systems}

\section{Integrated Modular Avionics}
This is about the IMA.

\section{ARINC 653}
\subsection{Overview}
ARINC 653 is a standard that was originally designed and created for the avionics industry.
It has been created to work within Integrated Modular Avionics,
(IMA) which required applications to be processed on single microprocessor module; so ARINC 653 was born.
ARINC 653 is said to “bring a new quality in real-time systems development“,
this is achieved through its API (APEX) which interfaces between the OS and the software applications.
The APEX manages partitions, memory allocation and error handling.
Applications are partitioned separately into those partitions with their own memory and scheduled processing time slots.
These partitions can not interfere with each other and can not take resources from each other.
If an error in a partition occurs, it will only crash itself; the remaining partitions will not be affected.

The idea is to bring such an operating system in to the wider embedded world.
To do that in a one semester project, the focus will be on understanding the standard,
working with embedded systems and implementing the parts of the operating system,
essential for running simple partitions on a single hardware unit.
\todo[inline]{Correct and expand on the introduction to what the standard is about.}

\subsection{All the details}

\section{Hardware}
In the world of embedded computing the computing platforms varies a lot more,
compared to the world of workstations and desktopcomputers.
Traditionally embedded computers are used for small specialized static tasks and
not required to run an operatingsystem with multiple processors.
For this reason many embedded systems are arranged as a SoC,
with limitied amounts of ram, flash and a low cpu clock,
not capable of supporting large computer screens and high speed networking.

Improved technology however, have brought powerful new CPUs of architectures used in the embedded world
such as MIPS and ARM.
The latter now being used in all levels of computing, from ultra low power single purpose embedded systems,
to workstations and supercomputer installations with graphics and highspeed networking requirements.
For more powerful computers this might represent both a shift from the widely used CISC architectured
to smaller RISC and a new focus on power efficiency over increased CPU speed.

For embedded devices this change has ment more capcable systems
and an increasingly narrowing gap between embedded and workstation computing.
Capable low power devices and cheap manufacturing of this new generation of RISC based computing systems,
has made it possible to build computers to solve problems on all scales using the same, or simular, platforms.
This has brought a surge development for the middle layer of computing,
a little more capable thán traditional embedded devices and a little less capable than the typical workstation.

This middle layer covers a wide range of devices with different capabilities.
Developing a system to operate in an embedded environment with real time requirements
and low overhead peripheral interaction,
yet with the flexibility and multitasking capabilities of a modern operating system,
.... bla bla bla
\missingfigure{We need a figure right here!}
\todo[inline,color=green]{Hardware analysis.}

\section{}

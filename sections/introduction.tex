\chapter{Introduction}\label{ch:introduction} 

Embedded devices are becoming more powerful and less expensive by the day. The large majority of
microprocessors produced end up as components of embedded devices, ranging from mobile phones
to traffic lights, from digital watches to medical apparatus. These devices are tailored for 
performing specific tasks, thus requiring the developers' skill and commitment to create reliable
systems. The diverse applications of these systems divide them into different categories, in regards to
how dependable they have to be. A wristwatch skipping a second has no severe consequences, while the
failure of a surgical robot has to be avoided at all costs. These requirements led to the creation 
of technical standards to be followed when developing such systems. 
\\\\
Some might think that these regulations are unfairly formulated and distributed, and in most
of the cases they might be right. For example, if we compare the existent standards for the aeronautical industry with the ones in the automotive field, there are quite big discrepancies. The standards 
in use in the automotive industry are applying certain rules from the avionics regulations, 
but only to a certain extent.
\todo {\url{<http://publica.fraunhofer.de/eprints/urn_nbn_de_0011-n-2485950.pdf>} - Can cars fly?}
However, as time goes, these two fields seem to converge in terms of safety requirements. 
In the past few years there has been done work for mapping and suitable integration 
of the norms from the aviation line to the automotive field. Just take the new self-driven 
cars developed by Tesla, and it becomes quite clear why there is need for stricter regulations. 
\\\\
These being said, it seems that there is a tendency and a need to migrate to the more strict standards. 
However, this entails more work in the sense of documentation, design consideration, testing and verification.
In this project we aim to look into this kind of development process, while adhering to a strict 
Standard.

This project's cornerstone is developing an Operating System, that would run safety critical 
applications. The two main areas of interest are:
\begin{itemize}
\item the Standard, that specifies how the system should operate
\item the physical platform that would contain the system
\end{itemize}
 The following section
\todo[inline,color=green]{Add reference to Analysis} 
provides more details about the software specification that needs to be fulfilled, as well as
the different available hardware options.
\\
In this project the ARINC 653 Standard is used. A typical application for this would be IMAs, which 
is a network of real-time computers in an aircraft. What makes them special, is the fact that they can
differentiate between the applications' criticality levels.

Our main goal is to implement ARINC 653 on an STM32 microcontroller.

The final product would be able to let developers and technicians work on safety critical systems,
 without having to have a deep understanding of the standard.
